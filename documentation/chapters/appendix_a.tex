\textbf{Sample 1:}\\ % java-python_GeeksForGeeks_4023_A_1
\textbf{Java Program}\\
\vspace*{-\baselineskip}
\begin{Verbatim}[frame=single]
// Java implementation to find  
// the sum of the given series 
import java.io.*; 
  
class GfG { 
      
// function to find the sum 
// of the given series 
static int sumOfTheSeries(int n) 
{ 
    // required sum 
    return (n * (n + 1) / 2) * 
            (2 * n + 1) / 3; 
} 
      
  
// Driver program to test above 
public static void main (String[] args)  
{ 
    int n = 5; 
      
    System.out.println("Sum = "+  
                sumOfTheSeries(n)); 
  
} 
  
} 
  
// This code is contributed by Gitanjali. 
\end{Verbatim}
\textbf{Python Program} \\
\vspace*{-\baselineskip}
\begin{Verbatim}[frame=single]
# Python3 implementation to find 
# the sum of the given series 
  
# functionn to find the sum 
# of the given series 
def sumOfTheSeries( n ): 
      
    # required sum 
    return int((n * (n + 1) / 2) *
            (2 * n + 1) / 3) 
              
# Driver program to test above 
n = 5
print("Sum =", sumOfTheSeries(n)) 
  
# This code is contributed by "Sharad_Bhardwaj". 
\end{Verbatim}
\newpage
\textbf{Sample 2:} \\ % java-python_ProjectEuler_p007_A_30
\textbf{Java Program} \\
\vspace*{-\baselineskip}
\begin{Verbatim}[frame=single, breaklines=true, breakanywhere=true]
/* 
 * Solution to Project Euler problem 7
 * Copyright (c) Project Nayuki. All rights reserved.
 * 
 * https://www.nayuki.io/page/project-euler-solutions
 * https://github.com/nayuki/Project-Euler-solutions
 */


public final class p007 implements EulerSolution {
	
	public static void main(String[] args) {
		System.out.println(new p007().run());
	}
	
	
	/* 
	 * Computers are fast, so we can implement this solution by testing each number
	 * individually for primeness, instead of using the more efficient sieve of Eratosthenes.
	 */
	public String run() {
		for (int i = 2, count = 0; ; i++) {
			if (Library.isPrime(i)) {
				count++;
				if (count == 10001)
					return Integer.toString(i);
			}
		}
	}
	
}
\end{Verbatim}
\textbf{Python Program} \\
\vspace*{-\baselineskip}
\begin{Verbatim}[frame=single, breaklines=true, breakanywhere=true]
# 
# Solution to Project Euler problem 7
# Copyright (c) Project Nayuki. All rights reserved.
# 
# https://www.nayuki.io/page/project-euler-solutions
# https://github.com/nayuki/Project-Euler-solutions
# 

import eulerlib, itertools


# Computers are fast, so we can implement this solution by testing each number
# individually for primeness, instead of using the more efficient sieve of Eratosthenes.
# 
# The algorithm starts with an infinite stream of incrementing integers starting at 2,
# filters them to keep only the prime numbers, drops the first 10000 items,
# and finally returns the first item thereafter.
def compute():
	ans = next(itertools.islice(filter(eulerlib.is_prime, itertools.count(2)), 10000, None))
	return str(ans)


if __name__ == "__main__":
	print(compute())
\end{Verbatim}
