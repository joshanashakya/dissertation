Program translation refers to the technical process of automatically converting the source code of a computer program written in one programming language into an equivalent program in another. Deep learning models like the transformer and Code Bidirectional Encoder Representations from Transformers (CodeBERT) models can be trained to perform such program translation. This study compares the transformer model and the CodeBERT-based encoder-decoder model on the program translation task. Specifically, it trains the 6 and 12-layer models for 50 and 100 epochs to translate programs written in Java to Python and Python to Java.
\\\\
A total of 3133 Java-Python parallel programs were collected, and then the models were trained using the preprocessed training data. To compare the models, the Bilingual Evaluation Understudy (BLEU) and CodeBLEU scores were calculated on the test dataset. Among different layered models, the transformer model with 6 layers trained for 50 epochs to translate from Java to Python achieved the highest BLEU and CodeBLEU scores, with values of 0.2812 and 0.2802, respectively. Similarly, the transformer model with 6 layers trained for 100 epochs to translate from Python to Java received the highest BLEU and CodeBLEU scores of 0.3891 and 0.4018, respectively.
\\\\
These results show that the transformer models perform better than the CodeBERT models. Also, the BLEU and CodeBLEU scores of the Java to Python and Python to Java translation models are different.
\\\\
\textbf{Keywords:} \textit{Machine Translation, Program Translation, Transformer, Code Bidirectional Encoder Representations from Transformers (CodeBERT), Bilingual Evaluation Understudy (BLEU), Code Bilingual Evaluation Understudy (CodeBLEU)}